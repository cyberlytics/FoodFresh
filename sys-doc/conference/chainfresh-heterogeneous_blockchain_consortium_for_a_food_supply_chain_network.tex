\documentclass[conference]{IEEEtran}
\IEEEoverridecommandlockouts

\usepackage{comment}
\usepackage{amsmath,amssymb,amsfonts}
\usepackage{algorithmic}
\usepackage{graphicx}
\usepackage[cmyk, table]{xcolor}
\usepackage{tikz}
\usetikzlibrary{arrows,arrows.meta}
\usepackage{pgfplots}

\usepackage{tabularx}
\usepackage{array}
\usepackage{booktabs}
\usepackage{threeparttable}
\usepackage{wasysym}
\usepackage{ragged2e}
\usepackage{enumitem}
\usepackage{textcomp}
\usepackage[nolist]{acronym}

\usepackage{hyperref}
\hypersetup{hidelinks}

\usepackage[style=ieee, backend=biber]{biblatex}
\addbibresource{bibliography.bib}

\def\BibTeX{{\rm B\kern-.05em{\sc i\kern-.025em b}\kern-.08em
    T\kern-.1667em\lower.7ex\hbox{E}\kern-.125emX}}

\graphicspath{ {./images/} }  
    
\begin{acronym}
\acro{babe}[BABE]{Blind Assignment of Blockchain Extension}
\acro{fscn}[FSCN]{Food Supply Chain Network}
\acro{grandpa}[GRANDPA]{GHOST-based Recursive Ancestor Deriving Prefix Agreement}
\acro{gui}[GUI]{Graphical User Interface}
\acro{ide}[IDE]{Integrated Development Environment}
\acro{ocw}[OCW]{Off-Chain Worker}
\acro{sc}[SC]{Supply Chain}
\acro{scm}[SCM]{Supply Chain Management}
\acro{scn}[SCN]{Supply Chain Network}
\acro{hrmp}[HRMP]{Horizontal Relay-routed Message Passing}
\acro{xcm}[XCM]{Cross-Chain Messaging}
\acro{xcmp}[XCMP]{Cross-Chain Message Passing}
\end{acronym}
    
\begin{document}

\title{FoodFresh: Multi-Chain Design for a Food Supply Chain Network}

\begin{comment}
\author{\IEEEauthorblockN{~}
\IEEEauthorblockA{\textit{~}\\
~ \\
~}
\and
\IEEEauthorblockN{~}
\IEEEauthorblockA{\textit{~}\\
~ \\
~}
}
\end{comment}

\maketitle

\begin{abstract}
We consider the problem of supply chain data visibility in a blockchain-enabled supply chain network. Existing methods typically record transactions happening in a supply chain on a single blockchain and are limited in their ability to deal with different levels of data visibility.
To address this limitation, we present FoodFresh - a multi-chain consortium were organizations store immutable data on their own blockchains. A decentralized hub coordinates the cross-chain exchange of digital assets among the heterogenous blockchains.  
Mechanisms for enabling blockchain interoperability help preserve the benefits of independent sovereign blockchains while allowing for sharing of data across blockchain boundaries.
\end{abstract}

\begin{IEEEkeywords}
blockchain; consortium; supply chain network; controlled transparency; interoperability
\end{IEEEkeywords}

\section{Introduction} \label{s:introduction}
The food industry comprises companies dedicated to manufacturing and processing raw materials and semi-finished products from agriculture, forestry, and fishing. In recent years, food supply chains have progressed from shorter, independent to more unified, coherent relationships among supply chain participants \cite{bourlakis2008food}. Developing long-term, and collaborative relationships requires evolutionary technological solutions to simultaneously retain a competitive edge. 

Blockchain technology is presented as a way to reduce fraud, increase supply chain visibility, and provide supply chain optimization. Current applications of blockchain technology in food supply chain management (e.g., IBM Food Trust) rely mainly on a single distributed ledger. The implications on supply chain networks are twofold: (i) organizations participating in multiple supply chains must share their data on multiple blockchains, and (ii) participants may see information originally not intended for them because all participants can view every transaction on a distributed ledger. 

In this paper, we propose FoodFresh - a multi-chain approach for supply chain networks, allowing organizations to store immutable data on their own blockchain. A decentralized hub coordinates the cross-chain communication among the heterogenous blockchains.  The hub further ensures that all parties comply to the overarching rules of the consortium. 

The remainder of the paper is organized as follows: in Section~\ref{s:related_work}, a selection of related work is presented. Subsequently, an overview of the relevant technology is given in Section~\ref{s:background}. Next, Section~\ref{s:chainfresh_approach} discusses our proposal with the design rationale. We conclude the paper in Section~\ref{s:conclusion}, followed by the references at the end. 

\section{Related work} \label{s:related_work}
Recently, various solutions for blockchain-enabled supply chains have been proposed. For instance, \citeauthor{longo2019blockchain} have presented a software connector to connect an Ethereum-like public blockchain with an enterprise information system \cite{longo2019blockchain}. The software connector allows companies to share information with their partners with different levels of visibility. \citeauthor{schulz2018multichain} \cite{schulz2018multichain} have proposed a blockchain-enabled distributed supply chain. Their main idea is a network-centric design, which incorporates domain-specific blockchains for handling specific business process and a hub or main blockchain that connects the blockchains to communicate with each other. 

Polkadot uses a hybrid consensus model, separating block production (\ac{babe}) from finality (\ac{grandpa}). This allows for blocks to be rapidly produced and finalized at a slower pace without risking slower transaction speeds or stalling.  Polkadot provides cross-chain communication with arbitrary data. Parachains communicate through the \ac{xcmp} protocol, a queuing communication mechanism based on a Merkle tree. \ac{xcmp} is designed to communicate arbitrary messages between parachains. Messages are sent together with the next parachain block (short: parablock), while the relay chain blocks include only the proof of post. All messages must be processed in a proper order, for which a chain of Merkle proofs is used. However, \ac{xcmp} is still under development. Therefore, the stop-gap protocol is \ac{hrmp}. As soon as \ac{xcmp} is fully developed, it can replace \ac{hrmp}. The primary difference between the two is the data stored on the relay chain. In \ac{hrmp}, the relay chain stores the full message with its payload. \ac{xcmp}, on the other hand, will only store a reference to the payload. The target parachain will be responsible for decoding the message payload.

\section{Background} \label{s:background}
This section provides a brief overview of the different relevant technologies: Section~ \ref{s:fscn} describes the characteristics of food supply chain networks, Section~ \ref{s:blockchain_types} presents different types of blockchain technology, and Section~ \ref{s:blockchain_interoperability} different blockchain interoperability approaches.

\subsection{Food Supply Chain Network} \label{s:fscn}
A supply chain is an interconnection of organizations, activities, resources, people, and information. Organizations along a food supply chain are dedicated to growing and processing raw materials (e.g., fruits) and semi-finished products (e.g., fruit juices) for delivery to the end customer. Food supply chains are complex and affected by various factors such as the sociopolitical environment \cite{van2005innovations}. Regulatory bodies such as the US Department of Agriculture (USDA) aim to protect consumer health and increase economic viability. Thus, they release frequent updates to ensure their criteria is met by food supply chains.

In an \ac{fscn}, more than one supply chain and more than one business process can be identified, both parallel and sequential in time. The parties involved in the business processes depend on the type of \ac{fscn}. This thesis considers an \ac{fscn} for fresh agricultural products.  

\citeauthor{van2005innovations} have identified farmers, retailers, and their logistics service suppliers as parties involved in an \ac{fscn} for fresh agricultural products \cite{van2005innovations}. Figure~\ref{fig:food_supply_network} depicts such a supply chain at the organization level within the context of an \ac{fscn} for fresh agricultural products. Each organization is positioned in a product lifecycle stage and belongs to at least one supply chain. That means an organization can have multiple suppliers and customers at the same time and over time. Figure~\ref{fig:food_supply_network} visualizes this by showing the perspective of the processor (bold lines), who has multiple connections to distributors and farmers. Other stakeholders such as nongovernmental organizations, governments, and shareholders are indirectly involved at each stage of the product lifecycle.

\begin{figure}[ht]
    \centering
    \includegraphics[page=1,scale=0.23]{fscn}
	\caption{Schematic diagram of an \ac{fscn} (based on \citeauthor{van2005innovations}~\cite{van2005innovations})}
	\label{fig:food_supply_network}
\end{figure}

\subsection{Blockchain Types} \label{s:blockchain_types}
There are three different types of blockchain systems \cite{zheng2017overview}. Public blockchains are considered permissionless because, in principle, everyone can attend the consensus process and read the stored data. The application of public blockchains has several use cases, including cryptocurrencies and document validation. In a consortium blockchain, an elected group of participants is allowed to attend the consensus process. The stored data may be read by selected members or by the public. Supply chain and research environments are two exemplary use cases for this sort of blockchain. In a private blockchain, all participants belong to the same organization, and the public cannot access the system. Two use cases for this final blockchain type are banking and asset ownership. Private and consortium blockchains are considered permissioned blockchains because, in both cases, only a limited group can attend the consensus process.

\subsection{Blockchain Interoperability} \label{s:blockchain_interoperability}
Blockchain interoperability involves the ability of independent distributed ledger networks to communicate with each other. Various approaches have been established to provide blockchain interoperability, resulting in a highly fragmented market \cite{belchior2021survey}.   were the first to conduct a systematic literature review on blockchain interoperability solutions. The Blockchain Interoperability Framework \citeauthor{belchior2021survey} categorizes interoperability solutions into three categories: interoperability across public blockchains (public connectors),  independent blockchains that interoperate among each other ( blockchains of blockchains), and finally approaches that neither fit into the public connectors nor blockchains of blockchains category (hybrid connector). 

\section{FoodFresh} \label{s:chainFresh}
In this section, we describe a consortium blockchain for a food supply chain network for the purpose of interoperability and controlled transparency. Section~\ref{s:chainfresh_approach}, introduces the approach. The three tiers of the system architecture are described in the following sections: the presentation tier in Section~\ref{s:presentation_tier},  the application tier in Section~\ref{s:application_tier}, and the relay tier in Section~\ref{s:relay_tier}.

\subsection{FoodFresh Approach} \label{s:chainfresh_approach}
The FoodFresh approach provides an implementation of the multi-chain approach (Section~ \ref{s:related_work}). The blockchain consortium comprises a multi-chain ecosystem for organizations. Each organization is allowed to participate in the consensus process. 
A permanent and shared record of food system data connects participants across the food supply chain network. This is done through the use of a main blockchain, called relay chain. The sole purpose of the relay chain is to coordinate and share appropriate data and ensure all parties are complying to overarching rules. Each organziation can setup and manage their own private blockchain, which keeps full control over the data to themselves. It also allows them to share immutable and accurate data easily to other participants in their supply network. This approach allows for the addition or removal of individual players from the ecosystem with minimal impact.

\subsection{System Architecture} \label{s:system_architecture}
FoodFresh as a distributed system is a composition of three tiers. 
This section will outline each of the three tiers. The presentation tier in Section~\ref{s:presentation_tier}, the application tier in Section~\ref{s:application_tier}, and finally the relay tier in Section~\ref{s:relay_tier}. Fig.~\ref{fig:system_architecture} depicts the system architecture for two interoperating supply chain organizations. 

\subsubsection{Presentation Tier} \label{s:presentation_tier}
To provide the user with convenient access to the FoodFresh system, the presentation tier is responsible for interacting with the application tier through a websocket connection. 
Any websocket-capable client or device can communicate with the endpoints exposed by the application tier. The user interacts with a \ac{gui} to manage the permissions of participating members, register shipments and products, and trace shipments along the supply chain. A browser extension is required to manage blockchain accounts and to sign transactions within those accounts.

\subsubsection{Application Tier} \label{s:application_tier}
The application tier encompasses application-specific blockchains (the parachains) that allow organizations to join with their own blockchain, where they can store immutable data. Through this, organizations are able to create products and shipments. A shipment's storage and transportation conditions can be monitored and tracked through the supply chain. Additionally, an \ac{ocw} is used to communicate the latest shipment status with the external world. With Cumulus, parachains are able to send and receive cross-chain messages and enable validators to validate their state transitions.

\begin{figure}[h!]
\centerline{\includegraphics[scale=0.6]{business_logic}}
\caption{An overview of the business logic, decomposed into pallets}
\label{fig:business_logic}
\end{figure}

\subsubsection{Relay Tier} \label{s:relay_tier}
The relay chain, in the relay tier, is the central hub in the network of heterogeneous blockchains, the parachains. The relay chain provides parachains with parablock validation and allows them to communicate with each other using the \ac{xcm} format for cross-chain messaging.

Validators are the actors of the relay chain and have three responsibilities: (1) to verify that the information contained in parablocks is valid such as the identities of the transacting parties, (2) to participate in the consensus mechanism to produce the relay chain blocks based on validity statements from other validators, and (3) to handle cross-chain messages. For validators to fulfill their responsibilities, they are equipped with six primary runtime modules. The \textit{inclusion} module handles the inclusion and availability of parablocks. In addition, \textit{shared} manages the shared storage and configurations for other validator modules. The \textit{paras} module manages the chain-head and validation code for parachains. The \textit{scheduler} is responsible for parachain scheduling as well as validator assignments for the consensus mechanism. The \textit{validity} module addresses secondary checks and disputes resolution for available parablocks. Finally, the \textit{\ac{xcmp}} module handles cross-chain messages and ensures that the messages are relayed to the receiving parachain. 

An integral part of cross-chain communication is the establishment of a cross-chain messaging channel between the validators of two communicating parachains. \citeauthor{burdges2020overview} \cite{burdges2020overview} have stated that a messaging channel aims to guarantee four things: ``First that messages arrive quickly; second that messages from one parachain arrive to another in order; third that arriving messages were indeed sent in the finalised history of the sending chain; and fourth that recipients will receive messages fairly across senders, helping guarantee that senders never wait indefinitely for their messages to be seen'' (section 4.4.3, p. 19).  

\begin{figure*}[h]
	\centering
	\includegraphics[width=\linewidth]{tiers_overview}
	\caption{Overview of our approach. The architecture is composed of three tiers: presentation, application, and relay. }
	\label{fig:system_architecture}
\end{figure*}

%Development
\subsection{Substrate Framework} \label{substrate_framework}
FoodFresh is built with substrate \cite{substrateio}, a modular framework for building blockchains. A nontechnical reason for using substrate is its flexibility. Organizations must be able to adapt their blockchain system to meet supply chain compliance requirements of regulatory bodies. Regulations happen frequently, especially in food supply chains, as shown in Section~\ref{s:fscn}. Due to the modular nature of substrate-based blockchains, developers have the necessary freedom to swap or add modules to their blockchain runtime.

Technical reasons include the chosen programming language, the software design, and the off-chain abilities. Substrate is implemented in the programming language Rust, which aims to provide performance (comparable to C++), reliability, and better means for productivity. In terms of reliability, Rust manages resources (including memory, files, network, and thread) and avoids problems such as resource leaks or data races. Finally, for productivity, Rust provides \ac{ide} support and type inspections. Furthermore, substrate is generic by design, meaning transactions are abstracted to extrinsics (things that happen outside the chain) and intrinsics (things that happen inside the chain). Transactions are stored as binary large objects. As a result, users can transfer and store any type of data on the blockchain. 

Nonetheless, with FoodFresh as a permissioned blockchain, concerns about off-chain processes need to be raised. For instance, \citeauthor{HELLIAR2020102136} have posited that ``off-chain processes may become a major barrier for permissioned blockchains'' \cite{HELLIAR2020102136}. Using substrate, off-chain data can be queried or processed before it is included in the on-chain state through \ac{ocw}, a collator node subsystem that allows for the execution of long-running and possibly nondeterministic tasks. Moreover, an \ac{ocw} does not influence the block production time.

\subsection{Deployment} \label{s:deployment}
FoodFresh requires validator nodes for  the relay chain and collator nodes for the parachains to be set up by the organizations participating in a supply chain network. Nodes can be deployed locally or remotely via a cloud service provider such as Amazon Web Services. 
Before parachains can participate in cross-chain communication, they need to be registered on the relay chain. The following rule is defined in the Collator Protocol \cite{collatorProtocol}, which implements the network protocol for the Collator-to-Validator networking: To accept $n$ parachain connections, $n + 1$ validator nodes need to run on the relay chain. For the FoodFresh prototype, two relay chain nodes are started to connect one parachain node. Further, the relay chain needs to obtain the hex-encoded parachain's genesis state (exported from a collator node) and the WebAssembly runtime validation function to validate parablocks.

\section{Conclusion and Future Work} \label{s:conclusion}
Developing long-term and increasingly collaborative relationships among supply chain participants requires advanced technological solutions to retain a competitive edge. Blockchain is presented as a promising technology that might increase supply chain visibility and improve efficiency.  We have presented FoodFresh - a multi-chain consortium for a food supply chain network. The design approach used for FoodFresh may benefit more than just the supply chain industry. In essence, it could apply to any network that requires the distribution or transfer of sensitive data. Future scope may be to apply the approach to other industries. 

% References
\renewcommand*{\UrlFont}{\rmfamily}
\printbibliography
\vspace{12pt}

\end{document}
